\newcommand{\NWtarget}[2]{#2}
\newcommand{\NWlink}[2]{#2}
\newcommand{\NWtxtMacroDefBy}{Fragment defined by}
\newcommand{\NWtxtMacroRefIn}{Fragment referenced in}
\newcommand{\NWtxtMacroNoRef}{Fragment never referenced}
\newcommand{\NWtxtDefBy}{Defined by}
\newcommand{\NWtxtRefIn}{Referenced in}
\newcommand{\NWtxtNoRef}{Not referenced}
\newcommand{\NWtxtFileDefBy}{File defined by}
\newcommand{\NWtxtIdentsUsed}{Uses:}
\newcommand{\NWtxtIdentsNotUsed}{Never used}
\newcommand{\NWtxtIdentsDefed}{Defines:}
\newcommand{\NWsep}{${\diamond}$}
\newcommand{\NWnotglobal}{(not defined globally)}
\newcommand{\NWuseHyperlinks}{}
%
% Copyright (c) 1996, Preston Briggs
% All rights reserved.
%
% Redistribution and use in source and binary forms, with or without
% modification, are permitted provided that the following conditions
% are met:
%
% Redistributions of source code must retain the above copyright notice,
% this list of conditions and the following disclaimer.
%
% Redistributions in binary form must reproduce the above copyright notice,
% this list of conditions and the following disclaimer in the documentation
% and/or other materials provided with the distribution.
%
% Neither name of the product nor the names of its contributors may
% be used to endorse or promote products derived from this software without
% specific prior written permission.
%
% THIS SOFTWARE IS PROVIDED BY THE COPYRIGHT HOLDERS AND CONTRIBUTORS
% ``AS IS'' AND ANY EXPRESS OR IMPLIED WARRANTIES, INCLUDING, BUT NOT
% LIMITED TO, THE IMPLIED WARRANTIES OF MERCHANTABILITY AND FITNESS
% FOR A PARTICULAR PURPOSE ARE DISCLAIMED. IN NO EVENT SHALL THE AUTHORS
% OR CONTRIBUTORS BE LIABLE FOR ANY DIRECT, INDIRECT, INCIDENTAL, SPECIAL,
% EXEMPLARY, OR CONSEQUENTIAL DAMAGES (INCLUDING, BUT NOT LIMITED TO,
% PROCUREMENT OF SUBSTITUTE GOODS OR SERVICES; LOSS OF USE, DATA, OR
% PROFITS; OR BUSINESS INTERRUPTION) HOWEVER CAUSED AND ON ANY THEORY
% OF LIABILITY, WHETHER IN CONTRACT, STRICT LIABILITY, OR TORT (INCLUDING
% NEGLIGENCE OR OTHERWISE) ARISING IN ANY WAY OUT OF THE USE OF THIS
% SOFTWARE, EVEN IF ADVISED OF THE POSSIBILITY OF SUCH DAMAGE.
%

% Notes:
% Update on 2011-12-16 from Keith Harwood.
% -- @s in scrap supresses indent of following fragment expansion
% -- @<Name@> in text is expanded
% Update on 2011-04-20 from Keith Harwood.
% -- @t provide fragment title in output 
% Changes from 2010-03-11 in the Sourceforge revision history. -- sjw
% Updates on 2004-02-23 from Gregor Goldbach
% <glauschwuffel@users.sourceforge.net>
% -- new command line option -r which will make nuweb typeset each
% NWtarget and NWlink instance with \LaTeX-commands from the hyperref
% package. This gives clickable scrap numbers which is very useful
% when viewing the document on-line.

% Updates on 2004-02-13 from Gregor Goldbach
% <glauschwuffel@users.sourceforge.net>
% -- new command line option -l which will make nuweb typeset scraps
%    with the help of LaTeX's listings packages instead or pure
%    verbatim.
% -- man page corrections and additions.

% Updates on 2003-04-24 from Keith Harwood <Keith.Harwood@vitalmis.com>
% -- sectioning commands (@s and global-plus-sign in scrap names)
% -- @x..@x labelling
% -- @f current file
% -- @\# suppress indent

% This file has been changed by Javier Goizueta <jgoizueta@jazzfree.es>
% on 2001-02-15.
% These are the changes:
% LANG  -- Introduction of \NW macros to substitue language dependent text
% DIAM  -- Introduction of \NWsep instead of the \diamond separator
% HYPER -- Introduction of hyper-references
% NAME  -- LaTeX formatting of macro names in HTML output
% ADJ   -- Adjust of the spacing between < and a fragment name
% TEMPN -- Fix of the use of tempnam
% LINE  -- Synchronizing #lines when @% is used
% MAC   -- definition of the macros used by LANG,DIAM,HYPER
% CHAR  -- Introduce @r to change the nuweb meta character (@)
% TIE   -- Replacement of ~ by "\nobreak\ "
% SCRAPs-- Elimination of s
% DNGL  -- Correction: option -d was not working and was misdocumented
%  --after the CHAR modifications, to be able to specify non-ascii characters
%    for the scape character, the program must be compiled with the -K
%    option in Borland compilers or the -funsigned-char in GNU's gcc
%    to treat char as an unsigned value when converted to int.
%    To make the program independent of those options, either char
%    should be changed to unsigned char (bad solution, since unsigned
%    char should be used for numerical purposes) or attention should
%    be payed to all char-int conversions. (including comparisons)

% --2002-01-15: the TILDE modificiation is necessary because some ties
%   have been introduced in version 0.93 in troublesome places when
%   the babel package is used with the spanish.ldf option (which makes
%   ~ an active character).

% --2002-01-15: an ``s'' was being added to the NWtxtDefBy and
%   NWtxtDefBy messages when followed by more than one reference.

\documentclass[a4paper]{report}
\newif\ifshowcode
\showcodetrue

\usepackage{latexsym}
%\usepackage{html}

\usepackage{listings}

\usepackage{color}
\definecolor{linkcolor}{rgb}{0, 0, 0.7}

\usepackage[%
backref,%
raiselinks,%
pdfhighlight=/O,%
pagebackref,%
hyperfigures,%
breaklinks,%
colorlinks,%
pdfpagemode=None,%
pdfstartview=FitBH,%
linkcolor={linkcolor},%
anchorcolor={linkcolor},%
citecolor={linkcolor},%
filecolor={linkcolor},%
menucolor={linkcolor},%
pagecolor={linkcolor},%
urlcolor={linkcolor}%
]{hyperref}

\setlength{\oddsidemargin}{0in}
\setlength{\evensidemargin}{0in}
\setlength{\topmargin}{0in}
\addtolength{\topmargin}{-\headheight}
\addtolength{\topmargin}{-\headsep}
\setlength{\textheight}{8.9in}
\setlength{\textwidth}{6.5in}
\setlength{\marginparwidth}{0.5in}

\title{Nuweb Version 1.58 \\ A Simple Literate Programming Tool}
\date{}
\author{Preston Briggs\thanks{This work has been supported by ARPA,
through ONR grant N00014-91-J-1989.}
\\ {\sl preston@tera.com}
\\ HTML scrap generator by John D. Ramsdell
\\ {\sl ramsdell@mitre.org}
\\ Scrap formatting by Marc W. Mengel
\\ {\sl mengel@fnal.gov}
\\ Continuing maintenance by Simon Wright
\\ {\sl simon@pushface.org}
\\ and Keith Harwood
\\ {\sl Keith.Harwood@vitalmis.com}}

\begin{document}
\pagenumbering{roman}
\maketitle
\tableofcontents

\chapter{Introduction}
\pagenumbering{arabic}

In 1984, Knuth introduced the idea of {\em literate programming\/} and
described a pair of tools to support the practise~\cite{Knuth:CJ-27-2-97}.
His approach was to combine Pascal code with \TeX\ documentation to
produce a new language, \verb|WEB|, that offered programmers a superior
approach to programming. He wrote several programs in \verb|WEB|,
including \verb|weave| and \verb|tangle|, the programs used to support
literate programming.
The idea was that a programmer wrote one document, the web file, that
combined documentation (written in \TeX~\cite{Knuth:ct-a}) with code
(written in Pascal).

Running \verb|tangle| on the web file would produce a complete
Pascal program, ready for compilation by an ordinary Pascal compiler.
The primary function of \verb|tangle| is to allow the programmer to
present elements of the program in any desired order, regardless of
the restrictions imposed by the programming language. Thus, the
programmer is free to present his program in a top-down fashion,
bottom-up fashion, or whatever seems best in terms of promoting
understanding and maintenance.

Running \verb|weave| on the web file would produce a \TeX\ file, ready
to be processed by \TeX\@. The resulting document included a variety of
automatically generated indices and cross-references that made it much
easier to navigate the code. Additionally, all of the code sections
were automatically prettyprinted, resulting in a quite impressive
document.

Knuth also wrote the programs for \TeX\ and {\small\sf METAFONT}
entirely in \verb|WEB|, eventually publishing them in book
form~\cite{Knuth:ct-b,Knuth:ct-d}. These are probably the
largest programs ever published in a readable form.

Inspired by Knuth's example, many people have experimented with
\verb|WEB|\@. Some people have even built web-like tools for their
own favorite combinations of programming language and typesetting
language. For example, \verb|CWEB|, Knuth's current system of choice,
works with a combination of C (or C++) and \TeX~\cite{Levy:TB8-1-12}.
Another system, FunnelWeb, is independent of any programming language
and only mildly dependent on \TeX~\cite{Williams:FUM92}. Inspired by the
versatility of FunnelWeb and by the daunting size of its
documentation, I decided to write my own, very simple, tool for
literate programming.%
\footnote{There is another system similar to
mine, written by Norman Ramsey, called {\em noweb}~\cite{Ramsey:LPT92}. It
perhaps suffers from being overly Unix-dependent and requiring several
programs to use. On the other hand, its command syntax is very nice.
In any case, nuweb certainly owes its name and a number of features to
his inspiration.}


\section{Nuweb}

Nuweb works with any programming language and \LaTeX~\cite{Lamport:LDP85}. I
wanted to use \LaTeX\ because it supports a multi-level sectioning
scheme and has facilities for drawing figures. I wanted to be able to
work with arbitrary programming languages because my friends and I
write programs in many languages (and sometimes combinations of
several languages), {\em e.g.,} C, Fortran, C++, yacc, lex, Scheme,
assembly, Postscript, and so forth. The need to support arbitrary
programming languages has many consequences:
\begin{description}
\item[No prettyprinting] Both \verb|WEB| and \verb|CWEB| are able to
  prettyprint the code sections of their documents because they
  understand the language well enough to parse it. Since we want to use
  {\em any\/} language, we've got to abandon this feature.
  However, we do allow particular individual formulas or fragments
  of \LaTeX\ code to be formatted and still be parts of output files.
  Also, keywords in scraps can be surrounded by \verb|@_| to
  have them be bold in the output.
\item[No index of identifiers] Because \verb|WEB| knows about Pascal,
  it is able to construct an index of all the identifiers occurring in
  the code sections (filtering out keywords and the standard type
  identifiers). Unfortunately, this isn't as easy in our case. We don't
  know what an identifier looks like in each language and we certainly
  don't know all the keywords. (On the other hand, see the end of
  Section~\ref{minorcommands})
\end{description}
Of course, we've got to have some compensation for our losses or the
whole idea would be a waste. Here are the advantages I can see:
\begin{description}
\item[Simplicity] The majority of the commands in \verb|WEB| are
  concerned with control of the automatic prettyprinting. Since we
  don't prettyprint, many commands are eliminated. A further set of
  commands is subsumed by \LaTeX\  and may also be eliminated. As a
  result, our set of commands is reduced to only four members (explained
  in the next section). This simplicity is also reflected in
  the size of this tool, which is quite a bit smaller than the tools
  used with other approaches.
\item[No prettyprinting] Everyone disagrees about how their code
  should look, so automatic formatting annoys many people. One approach
  is to provide ways to control the formatting. Our approach is
  simpler---we perform no automatic formatting and therefore allow the
  programmer complete control of code layout.
  We do allow individual scraps to be presented in either verbatim,
  math, or paragraph mode in the \TeX\ output.
\item[Control] We also offer the programmer complete control of the
  layout of his output files (the files generated during tangling). Of
  course, this is essential for languages that are sensitive to layout;
  but it is also important in many practical situations, {\em e.g.,}
  debugging.
\item[Speed] Since nuweb doesn't do too much, the nuweb tool runs
  quickly. I combine the functions of \verb|tangle| and \verb|weave| into
  a single program that performs both functions at once.
\item[Page numbers] Inspired by the example of noweb, nuweb refers to
  all scraps by page number to simplify navigation. If there are
  multiple scraps on a page (say, page~17), they are distinguished by
  lower-case letters ({\em e.g.,} 17a, 17b, and so forth).
\item[Multiple file output] The programmer may specify more than one
  output file in a single nuweb file. This is required when constructing
  programs in a combination of languages (say, Fortran and C)\@. It's also
  an advantage when constructing very large programs that would require
  a lot of compile time.
\end{description}
This last point is very important. By allowing the creation of
multiple output files, we avoid the need for monolithic programs.
Thus we support the creation of very large programs by groups of
people.

A further reduction in compilation time is achieved by first
writing each output file to a temporary location, then comparing the
temporary file with the old version of the file. If there is no
difference, the temporary file can be deleted. If the files differ,
the old version is deleted and the temporary file renamed. This
approach works well in combination with \verb|make| (or similar tools),
since \verb|make| will avoid recompiling untouched output files.

\subsection{Nuweb and HTML}

In addition to producing {\LaTeX} source, nuweb can be used to
generate HyperText Markup Language (HTML), the markup language used by
the World Wide Web.  HTML provides hypertext links.  When an HTML
document is viewed online, a user can navigate within the document by
activating the links.  The tools which generate HTML automatically
produce hypertext links from a nuweb source.

(Note that hyperlinks can be included in \LaTeX\ using the
\verb|hyperref| package. This is now the preferred way of doing
this and the HTML processing is not up to date.)

\section{Writing Nuweb}

The bulk of a nuweb file will be ordinary \LaTeX\@. In fact, any
{\LaTeX} file can serve as input to nuweb and will be simply copied
through, unchanged, to the documentation file---unless a nuweb command
is discovered. All nuweb commands begin with an ``at-sign''
(\verb|@|).  Therefore, a file without at-signs will be copied
unchanged.  Nuweb commands are used to specify {\em output files,}
define {\em fragments,} and delimit {\em scraps}. These are the basic
features of interest to the nuweb tool---all else is simply text to be
copied to the documentation file.

\subsection{The Major Commands}

Files and fragments are defined with the following commands:
\begin{description}
\item[{\tt @o} {\em file-name flags scrap\/}] Output a file. The file
  name is terminated by whitespace.
\item[{\tt @d} {\em fragment-name scrap\/}] Define a fragment. The
  fragment name is terminated by a return or the beginning of a scrap.
\item[{\tt @q} {\em fragment-name scrap\/}] Define a fragment. The
  fragment name is terminated by a return or the beginning of a scrap.
  This a quoted fragment.
\end{description}
A specific file may be specified several times, with each definition
being written out, one after the other, in the order they appear.
The definitions of fragments may be similarly specified piecemeal.

A fragment name may have embedded parameters. The parameters are
denoted by the sequence \texttt{@'value@'} where \texttt{value} is
an uninterpreted string of characters (although the sequence
\texttt{@@} denotes a single \texttt{@} character). When a fragment
name is used inside a scrap the parameters may be replaced by an
argument which may be a different literal string, a fragment use, an
embedded fragment or by a parameter use.

The difference between a quoted fragment (\texttt{@q}) and an
ordinary one (\texttt{@d}) is that inside a quoted fragment fragments
are not expanded on output. Rather, they are formatted as uses of
fragments so that the output file can itself be \texttt{nuweb}
source. This allows you to create files containing fragments which can
undergo further processing before the fragments are expanded, while
also describing and documenting them in one place.

You can have both quoted and unquoted fragments with the same
name. They are written out in order as usual, with those introduced by
\texttt{@q} being quoted and those with \texttt{@d} expanded as
normal.

In quoted fragments, the \texttt{@f} filename is quoted as well, so that
when it is expanded it refers to the finally produced file, not
any of the intermediate ones.

\subsubsection{Scraps}

Scraps have specific begin markers and end markers to allow precise
control over the contents and layout. Note that any amount of
whitespace (including carriage returns) may appear between a name and
the beginning of a scrap. Scraps may also appear in the running text
where they are formatted (almost) identically to their use in definitions,
but don't appear in any code output.
\begin{description}
\item[\tt @\{{\em anything\/}@\}] where the scrap body includes every
  character in {\em anything\/}---all the blanks, all the tabs, all the
  carriage returns.  This scrap will be typeset in verbatim mode. Using
  the \texttt{-l} option will cause the program to typeset the scrap with
  the help of \LaTeX's \texttt{listings} package.
\item[\tt @[{\em anything\/}@]] where the scrap body includes every
  character in {\em anything\/}---all the blanks, all the tabs, all the
  carriage returns.  This scrap will be typeset in paragraph mode, allowing
  sections of \TeX\ documents to be scraps, but still  be prettyprinted
  in the document.
\item[\tt @({\em anything\/}@)] where the scrap body includes every
  character in {\em anything\/}---all the blanks, all the tabs, all the
  carriage returns.  This scrap will be typeset in math mode.  This allows
  this scrap to contain a formula which will be typeset nicely.
\end{description}
Inside a scrap, we may invoke a fragment.
\begin{description}
\item[\tt @<{\em fragment-name\/}@>]
  This is a fragment use. It causes the fragment
  {\em fragment-name\/} to be expanded inline as the code is written out
  to a file. It is an error to specify recursive fragment invocations.
\item[\tt @<{\em fragment-name\/}@( {\em a1} @, {\em a2} @) @>] This
  is the old form of parameterising a fragment. It causes the fragment
  {\em fragment-name\/} to be expanded inline with the arguments {\em a1},
    {\em a2}, etc. Up to 9 arguments may be given.
\item[\tt @1, @2, ..., @9] In a fragment causes the n'th fragment
      argument to be substituted into the scrap.  If the argument
      is not passed, a null string is substituted.
      Arguments can be passed in two ways, either embedded in the
      fragment
      name or as the old form given above.

      An embedded argument may specified in four ways.
      \begin{description}
      \item[\tt @'string@']
      The string will be copied literally into the called fragment.
      It will not be interpreted (except for \texttt{@@} converted
            to \texttt{@}).
      \item[\tt @<{\em fragment-name\/}@>]
      The fragment will be expanded in the usual fashion and the results
      passed to the called fragment.
      \item[\tt @\{text@\}]
      The text will be expanded as normal and the results passed to
      the called fragment. This behaves like an anonymous fragment which has
      the same arguments as the calling fragment. Its principle use is
      to combine text and arguments into one argument.
      \item[\tt @1, @2, ..., @9]
      The argument of the calling fragment will be passed to the called
      fragment and expanded in there.
      \end{description}

      If an argument is used but there is no corresponding parameter
      in the fragment name, the null string is substituted. But what
      happens if there is a parameter in the full name of a fragment,
      but a particular application of the fragment is abbreviated
      (using the \verb|. . .| notation) and the argument is missed? In
      that case the argument is replaced by the string given in the
      definition of the fragment.

      In the old form the parameter may contain any text and will be
      expanded as a normal scrap. The two forms of parameter passing
      don't play nicely together. If a scrap passes both embedded and
      old form arguments the old form arguments are ignored.
\item[\tt @x{\em label}@x] Marks a place inside a scrap and
associates it to the label (which can be any text not containing
a @). Expands to the reference number of the scrap followed by a
numeric value. Outside scraps it expands to the same value. It is
used so that text outside scraps can refer to particular places
within scraps.
\item[\tt @f] Inside a scrap this is replaced by the name of the
current output file.
\item[\tt @t] Inside a scrap this is replaced by the title of the
fragment as it is at the point it is used, with all parameters
replaced by actual arguments.
\item[\tt @\#] At the beginning of a line in a scrap this will
suppress the normal indentation for that line. Use this, for
example, when you have a \verb|#ifdef| inside a nested scrap.
Writing \verb|@##ifdef| will cause it to be lined up on the left
rather than indented with the rest of its code.
\item[\tt @s] Inside a scrap supresses indentation for the following
fragment expansion.
\end{description}
Note that fragment names may be abbreviated, either during invocation or
definition. For example, it would be very tedious to have to
type, repeatedly, the fragment name
\begin{quote}
\verb|@d Check for terminating at-sequence and return name if found|
\end{quote}
Therefore, we provide a mechanism (stolen from Knuth) of indicating
abbreviated names.
\begin{quote}
\verb|@d Check for terminating...|
\end{quote}
Basically, the programmer need only type enough characters to
identify the fragment name uniquely, followed by three periods. An abbreviation
may even occur before the full version; nuweb simply preserves the
longest version of a fragment name. Note also that blanks and tabs are
insignificant within a fragment name; each string of them is replaced by a
single blank.

Sometimes, for instance during program testing, it is convenient to comment
out a few lines of code. In C or Fortran placing \verb|/* ... */| around the relevant
code is not a robust solution, as the code itself may contain
comments. Nuweb provides the command
\begin{quote}
\verb|@%|
\end{quote}only to be used inside scraps. It behaves exactly the same
as \verb|%| in the normal {\LaTeX} text body.

When scraps are written to a program file or a documentation file, tabs are
expanded into spaces by default. Currently, I assume tab stops are set
every eight characters. Furthermore, when a fragment is expanded in a scrap,
the body of the fragment is indented to match the indentation of the
fragment invocation. Therefore, care must be taken with languages
({\em e.g.,} Fortran) that are sensitive to indentation.
These default behaviors may be changed for each output file (see
below).

\subsubsection{Flags}

When defining an output file, the programmer has the option of using
flags to control output of a particular file. The flags are intended
to make life a little easier for programmers using certain languages.
They introduce little language dependences; however, they do so only
for a particular file. Thus it is still easy to mix languages within a
single document. There are four ``per-file'' flags:
\begin{description}
\item[\tt -d] Forces the creation of \verb|#line| directives in the
  output file. These are useful with C (and sometimes C++ and Fortran) on
  many Unix systems since they cause the compiler's error messages to
  refer to the web file rather than to the output file. Similarly, they
  allow source debugging in terms of the web file.
\item[\tt -i] Suppresses the indentation of fragments. That is, when a
  fragment is expanded within a scrap, it will {\em not\/} be indented to
  match the indentation of the fragment invocation. This flag would seem
  most useful for Fortran programmers.
\item[\tt -t] Suppresses expansion of tabs in the output file. This
  feature seems important when generating \verb|make| files.
\item[\tt -c{\it x}] Puts comments in generated code documenting the
fragment that generated that code. The {\it x} selects the comment style
for the language in use. So far the only valid values are \verb|c| to get
\verb|C| comment style, \verb|+| for \verb|C++| and \verb|p| for
\verb|Perl|. (\verb|Perl| commenting can be used for several languages
including \verb|sh| and, mostly, \verb|tcl|.)
If the global \verb|-x| cross-reference flag is
set the comment includes the page reference for the first scrap that
generated the code.
\end{description}


\subsection{The Minor Commands\label{minorcommands}}

We have some very low-level utility commands that may appear anywhere
in the web file.
\begin{description}
\item[\tt @@] Causes a single ``at sign'' to be copied into the output.
\item[\tt @\_] Causes the text between it and the next {\tt @\_}
      to be made bold (for keywords, etc.)
\item[\tt @i {\em file-name\/}] Includes a file. Includes may be
  nested, though there is currently a limit of 10~levels. The file name
  should be complete (no extension will be appended) and should be
  terminated by a carriage return. Normally the current directory is
  searched for the file to be included, but this can be varied using
  the \verb|-I| flag on the command line. Each such flag adds one
  directory tothe search path and they are searched in the order
  given.
\item[{\tt @r}$x$] Changes the escape character `@' to `$x$'.
  This must appear before any scrap definitions.
\item[\tt @v] Always replaced by the string established by
the \texttt{-V} flag, or a default string if the flag isn't
given. This is intended to mark versions of the generated
files.
\end{description}

In the text of the document, that is outside scraps, you may include
scrap-like material.

\begin{description}
\item[\tt @\{\textit{Anything}@\}]
The included material, the \textit{Anything},
is typeset as if it appeared inside a scrap. This is useful for
referring to fragments in the text and for
describing the literate programming process itself.
\item[\tt @<\textit{Fragment name}@>]
The fragment named is expanded in place in the text. 
The expansion is presented verbatim, it is not interpretted for
typesetting, so any special environment must be set up before and
after this is used.
\end{description}

Finally, there are three commands used to create indices to the
fragment
names, file definitions, and user-specified identifiers.
\begin{description}
\item[\tt @f] Create an index of file names.
\item[\tt @m] Create an index of fragment names.
\item[\tt @u] Create an index of user-specified identifiers.
\end{description}
I usually put these in their own section
in the \LaTeX\ document; for example, see Chapter~\ref{indices}.

Identifiers must be explicitly specified for inclusion in the
\verb|@u| index. By convention, each identifier is marked at the
point of its definition; all references to each identifier (inside
scraps) will be discovered automatically. To ``mark'' an identifier
for inclusion in the index, we must mention it at the end of a scrap.
For example,
\begin{quote}
\begin{verbatim}
@d a scrap @{
Let's pretend we're declaring the variables FOO and BAR
inside this scrap.
@| FOO BAR @}
\end{verbatim}
\end{quote}
I've used alphabetic identifiers in this example, but any string of
characters (not including whitespace or \verb|@| characters) will do.
Therefore, it's possible to add index entries for things like
\verb|<<=| if desired. An identifier may be declared in more than one
scrap.

In the generated index, each identifier appears with a list of all the
scraps using and defining it, where the defining scraps are
distinguished by underlining. Note that the identifier doesn't
actually have to appear in the defining scrap; it just has to be in
the list of definitions at the end of a scrap.

\section{Sectioning commands}
For larger documents the indexes and usage lists get rather
unwieldy and problems arise in naming things so that different
things in different parts of the document don't get confused. We
have a sectioning command which keeps the fragment names and user
identifiers separate. Thus, you can give a fragment in one section
the same name as a fragment in another and the two won't be confused
or connected in any way. Nor will user identifiers
defined in one section be referenced in another. Except for the
fact that scraps in successive sections can go into the same
output file, this is the same as if the sections came from
separate input files.

However, occasionally you may really want fragments from one section
to be used in another. More often, you will want to identify a
user identifier in one section with the same identifier in
another (as, for example, a header file defined in one section is
included in code in another). Extra commands allow a fragment
defined in one section to be accessible from all other sections.
Similarly, you can have scraps which define user identifiers and
export them so that they can be used in other sections.

\begin{description}
\item[{\tt @s}] Start a new section.
\item[{\tt @S}] Close the current section and don't start another.
\item[{\tt @d+} fragment-name scrap] Define a fragment which is
accessible in all sections, a global fragment.
\item[{\tt @D+}] Likewise
\item[{\tt @q+}] Likewise
\item[{\tt @Q+}] Likewise
\item[{\tt @m+}] Create an index of all such fragments.
\item[{\tt @u+}] Create an index of globally accessible
user identifiers.
\end{description}

There are two kinds of section, the base section which is where
normally everything goes, and local sections which are introduced with
the {\tt @s} command. A local section comprises everything from the
command which starts it to the one which ends it. A {\tt @s} command
will start a new local section. A {\tt @S} command closes the current
local section, but doesn't open another, so what follows goes into the
base section. Note that fragments defined in the base section aren't
global; they are accessible only in the base section, but they are
accessible regardless of any local sections between their definition
and their use.

Within a scrap:
\begin{description}
\item[\tt @<+{\em fragment-name\/}@>]
Expand the globally accessible fragment with that name, rather than
any local fragment.
\item[{\tt @+}] Like \verb"@|" except that the identifiers
defined are exported to the global realm and are not directly
referenced in any scrap in any section (not even the one where
they are defined).
\item[{\tt @-}] Like \verb"@|" except that the identifiers
are imported to the local realm. The cross-references show where
the global variables are defined and defines the same names as
locally accesible. Uses of the names within the section will
point to this scrap.
\end{description}

Note that the \verb"+" signs above are part of the commands. They
are not part of the fragment names. If you want a fragment whose name
begins with a plus sign, leave a space between the command and the
name.

\section{Running Nuweb}

Nuweb is invoked using the following command:
\begin{quote}
{\tt nuweb} {\em flags file-name}\ldots
\end{quote}
One or more files may be processed at a time. If a file name has no
extension, \verb|.w| will be appended.  {\LaTeX} suitable for
translation into HTML by {\LaTeX}2HTML will be produced from
files whose name ends with \verb|.hw|, otherwise, ordinary {\LaTeX} will be
produced.  While a file name may specify a file in another directory,
the resulting documentation file will always be created in the current
directory. For example,
\begin{quote}
{\tt nuweb /foo/bar/quux}
\end{quote}
will take as input the file \verb|/foo/bar/quux.w| and will create the
file \verb|quux.tex| in the current directory.

By default, nuweb performs both tangling and weaving at the same time.
Normally, this is not a bottleneck in the compilation process;
however, it's possible to achieve slightly faster throughput by
avoiding one or another of the default functions using command-line
flags. There are currently three possible flags:
\begin{description}
\item[\tt -t] Suppress generation of the documentation file.
\item[\tt -o] Suppress generation of the output files.
\item[\tt -c] Avoid testing output files for change before updating them.
\end{description}
Thus, the command
\begin{quote}
\verb|nuweb -to /foo/bar/quux|
\end{quote}
would simply scan the input and produce no output at all.

There are several additional command-line flags:
\begin{description}
\item[\tt -v] For ``verbose'', causes nuweb to write information about
  its progress to \verb|stderr|.
\item[\tt -n] Forces scraps to be numbered sequentially from~1
  (instead of using page numbers). This form is perhaps more desirable
  for small webs.
\item[\tt -s] Doesn't print list of scraps making up each file
  following each scrap.
\item[\tt -d] Print ``dangling'' identifiers -- user identifiers which
  are never referenced, in indices, etc.
\item[\tt -p {\it path}] Prepend \textit{path} to the filenames for
all the output files.
\item[\tt -l] \label{sec:pretty-print} Use the \texttt{listings}
package for formatting scraps. Use this if you want to have a
pretty-printer for your scraps. In order to e.g. have pretty Perl
scraps, include the following \LaTeX\ commands in your document:
\lstset{language=[LaTeX]TeX}

\begin{lstlisting}{language={[LaTeX]TeX}}
\usepackage{listings}
...
\lstset{extendedchars=true,keepspaces=true,language=perl}
\end{lstlisting}

See the \texttt{listings} documentation for a list of formatting
options. Be sure to include a \texttt{\char92
usepackage\{listings\}}
in your document.

\item[\tt -V \it string] Provide \textit{string} as the
replacement for the @v operation.
\end{description}

\section{Generating HTML}

Nikos Drakos' {\LaTeX}2HTML Version 0.5.3~\cite{drakos:94} can be used
to translate {\LaTeX} with embedded HTML scraps into HTML\@.  Be sure
to include the document-style option \verb|html| so that {\LaTeX} will
understand the hypertext commands.  When translating into HTML, do not
allow a document to be split by specifying ``\verb|-split 0|''.
You need not generate navigation links, so also specify
``\verb|-no_navigation|''.

While preparing a web, you may want to view the program's scraps without
taking the time to run {\LaTeX}2HTML\@.  Simply rename the generated
{\LaTeX} source so that its file name ends with \verb|.html|, and view
that file.  The documentations section will be jumbled, but the
scraps will be clear.

(Note that the HTML generation is not currently maintained. If the
only reason you want HTML is ti get hyperlinks, use the {\LaTeX}
\verb|hyperref| package and produce your document as PDF via
\verb|pdflatex|.)

\section{Restrictions}

Because nuweb is intended to be a simple tool, I've established a few
restrictions. Over time, some of these may be eliminated; others seem
fundamental.
\begin{itemize}
\item The handling of errors is not completely ideal. In some cases, I
  simply warn of a problem and continue; in other cases I halt
  immediately. This behavior should be regularized.
\item I warn about references to fragments that haven't been defined, but
  don't halt. The name of the fragment is included in the output file
  surrounded by \verb|<>| signs.
  This seems most convenient for development, but may change
  in the future.
\item File names and index entries should not contain any \verb|@|
  signs.
\item Fragment names may be (almost) any well-formed \TeX\ string.
  It makes sense to change fonts or use math mode; however, care should
  be taken to ensure matching braces, brackets, and dollar signs.
  When producing HTML, fragments are displayed in a preformatted element
  (PRE), so fragments may contain one or more A, B, I, U, or P elements
  or data characters.
\item Anything is allowed in the body of a scrap; however, very
  long scraps (horizontally or vertically) may not typeset well.
\item Temporary files (created for comparison to the eventual
  output files) are placed in the current directory. Since they may be
  renamed to an output file name, all the output files should be on the
  same file system as the current directory. Alternatively, you can
  use the \verb|-p| flag to specify where the files go.
\item Because page numbers cannot be determined until the document has
  been typeset, we have to rerun nuweb after \LaTeX\ to obtain a clean
  version of the document (very similar to the way we sometimes have
  to rerun \LaTeX\ to obtain an up-to-date table of contents after
  significant edits).  Nuweb will warn (in most cases) when this needs
  to be done; in the remaining cases, \LaTeX\ will warn that labels
  may have changed.
\end{itemize}
Very long scraps may be allowed to break across a page if declared
with \verb|@O| or \verb|@D| (instead of \verb|@o| and \verb|@d|).
This doesn't work very well as a default, since far too many short
scraps will be broken across pages; however, as a user-controlled
option, it seems very useful.  No distinction is made between the
upper case and lower case forms of these commands when generating
HTML\@.

\section{Acknowledgements}

Several people have contributed their times, ideas, and debugging
skills. In particular, I'd like to acknowledge the contributions of
Osman Buyukisik, Manuel Carriba, Adrian Clarke, Tim Harvey, Michael
Lewis, Walter Ravenek, Rob Shillingsburg, Kayvan Sylvan, Dominique
de~Waleffe, and Scott Warren.  Of course, most of these people would
never have heard or nuweb (or many other tools) without the efforts of
George Greenwade.

Since maintenance has been taken over by Marc Mengel, Simon Wright and
Keith Harwood online contributions have been made by:
\begin{itemize}
\item Walter Brown \verb|<wb@fnal.gov>|
\item Nicky van Foreest \verb|<n.d.vanforeest@math.utwente.nl>|
\item Javier Goizueta \verb|<jgoizueta@jazzfree.com>|
\item Alan Karp \verb|<karp@hp.com>|
\end{itemize}


\bibliographystyle{amsplain}
\bibliography{litprog,master,misc}

\end{document}
